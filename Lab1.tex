
% Default to the notebook output style

    


% Inherit from the specified cell style.




    
\documentclass{article}

    
    
    \usepackage{graphicx} % Used to insert images
    \usepackage{adjustbox} % Used to constrain images to a maximum size 
    \usepackage{color} % Allow colors to be defined
    \usepackage{enumerate} % Needed for markdown enumerations to work
    \usepackage{geometry} % Used to adjust the document margins
    \usepackage{amsmath} % Equations
    \usepackage{amssymb} % Equations
    \usepackage{eurosym} % defines \euro
    \usepackage[mathletters]{ucs} % Extended unicode (utf-8) support
    \usepackage[utf8x]{inputenc} % Allow utf-8 characters in the tex document
    \usepackage{fancyvrb} % verbatim replacement that allows latex
    \usepackage{grffile} % extends the file name processing of package graphics 
                         % to support a larger range 
    % The hyperref package gives us a pdf with properly built
    % internal navigation ('pdf bookmarks' for the table of contents,
    % internal cross-reference links, web links for URLs, etc.)
    \usepackage{hyperref}
    \usepackage{longtable} % longtable support required by pandoc >1.10
    \usepackage{booktabs}  % table support for pandoc > 1.12.2
    

    
    
    \definecolor{orange}{cmyk}{0,0.4,0.8,0.2}
    \definecolor{darkorange}{rgb}{.71,0.21,0.01}
    \definecolor{darkgreen}{rgb}{.12,.54,.11}
    \definecolor{myteal}{rgb}{.26, .44, .56}
    \definecolor{gray}{gray}{0.45}
    \definecolor{lightgray}{gray}{.95}
    \definecolor{mediumgray}{gray}{.8}
    \definecolor{inputbackground}{rgb}{.95, .95, .85}
    \definecolor{outputbackground}{rgb}{.95, .95, .95}
    \definecolor{traceback}{rgb}{1, .95, .95}
    % ansi colors
    \definecolor{red}{rgb}{.6,0,0}
    \definecolor{green}{rgb}{0,.65,0}
    \definecolor{brown}{rgb}{0.6,0.6,0}
    \definecolor{blue}{rgb}{0,.145,.698}
    \definecolor{purple}{rgb}{.698,.145,.698}
    \definecolor{cyan}{rgb}{0,.698,.698}
    \definecolor{lightgray}{gray}{0.5}
    
    % bright ansi colors
    \definecolor{darkgray}{gray}{0.25}
    \definecolor{lightred}{rgb}{1.0,0.39,0.28}
    \definecolor{lightgreen}{rgb}{0.48,0.99,0.0}
    \definecolor{lightblue}{rgb}{0.53,0.81,0.92}
    \definecolor{lightpurple}{rgb}{0.87,0.63,0.87}
    \definecolor{lightcyan}{rgb}{0.5,1.0,0.83}
    
    % commands and environments needed by pandoc snippets
    % extracted from the output of `pandoc -s`
    \providecommand{\tightlist}{%
      \setlength{\itemsep}{0pt}\setlength{\parskip}{0pt}}
    \DefineVerbatimEnvironment{Highlighting}{Verbatim}{commandchars=\\\{\}}
    % Add ',fontsize=\small' for more characters per line
    \newenvironment{Shaded}{}{}
    \newcommand{\KeywordTok}[1]{\textcolor[rgb]{0.00,0.44,0.13}{\textbf{{#1}}}}
    \newcommand{\DataTypeTok}[1]{\textcolor[rgb]{0.56,0.13,0.00}{{#1}}}
    \newcommand{\DecValTok}[1]{\textcolor[rgb]{0.25,0.63,0.44}{{#1}}}
    \newcommand{\BaseNTok}[1]{\textcolor[rgb]{0.25,0.63,0.44}{{#1}}}
    \newcommand{\FloatTok}[1]{\textcolor[rgb]{0.25,0.63,0.44}{{#1}}}
    \newcommand{\CharTok}[1]{\textcolor[rgb]{0.25,0.44,0.63}{{#1}}}
    \newcommand{\StringTok}[1]{\textcolor[rgb]{0.25,0.44,0.63}{{#1}}}
    \newcommand{\CommentTok}[1]{\textcolor[rgb]{0.38,0.63,0.69}{\textit{{#1}}}}
    \newcommand{\OtherTok}[1]{\textcolor[rgb]{0.00,0.44,0.13}{{#1}}}
    \newcommand{\AlertTok}[1]{\textcolor[rgb]{1.00,0.00,0.00}{\textbf{{#1}}}}
    \newcommand{\FunctionTok}[1]{\textcolor[rgb]{0.02,0.16,0.49}{{#1}}}
    \newcommand{\RegionMarkerTok}[1]{{#1}}
    \newcommand{\ErrorTok}[1]{\textcolor[rgb]{1.00,0.00,0.00}{\textbf{{#1}}}}
    \newcommand{\NormalTok}[1]{{#1}}
    
    % Define a nice break command that doesn't care if a line doesn't already
    % exist.
    \def\br{\hspace*{\fill} \\* }
    % Math Jax compatability definitions
    \def\gt{>}
    \def\lt{<}
    % Document parameters
    \title{Lab1}
    
    
    

    % Pygments definitions
    
\makeatletter
\def\PY@reset{\let\PY@it=\relax \let\PY@bf=\relax%
    \let\PY@ul=\relax \let\PY@tc=\relax%
    \let\PY@bc=\relax \let\PY@ff=\relax}
\def\PY@tok#1{\csname PY@tok@#1\endcsname}
\def\PY@toks#1+{\ifx\relax#1\empty\else%
    \PY@tok{#1}\expandafter\PY@toks\fi}
\def\PY@do#1{\PY@bc{\PY@tc{\PY@ul{%
    \PY@it{\PY@bf{\PY@ff{#1}}}}}}}
\def\PY#1#2{\PY@reset\PY@toks#1+\relax+\PY@do{#2}}

\expandafter\def\csname PY@tok@gd\endcsname{\def\PY@tc##1{\textcolor[rgb]{0.63,0.00,0.00}{##1}}}
\expandafter\def\csname PY@tok@gu\endcsname{\let\PY@bf=\textbf\def\PY@tc##1{\textcolor[rgb]{0.50,0.00,0.50}{##1}}}
\expandafter\def\csname PY@tok@gt\endcsname{\def\PY@tc##1{\textcolor[rgb]{0.00,0.27,0.87}{##1}}}
\expandafter\def\csname PY@tok@gs\endcsname{\let\PY@bf=\textbf}
\expandafter\def\csname PY@tok@gr\endcsname{\def\PY@tc##1{\textcolor[rgb]{1.00,0.00,0.00}{##1}}}
\expandafter\def\csname PY@tok@cm\endcsname{\let\PY@it=\textit\def\PY@tc##1{\textcolor[rgb]{0.25,0.50,0.50}{##1}}}
\expandafter\def\csname PY@tok@vg\endcsname{\def\PY@tc##1{\textcolor[rgb]{0.10,0.09,0.49}{##1}}}
\expandafter\def\csname PY@tok@m\endcsname{\def\PY@tc##1{\textcolor[rgb]{0.40,0.40,0.40}{##1}}}
\expandafter\def\csname PY@tok@mh\endcsname{\def\PY@tc##1{\textcolor[rgb]{0.40,0.40,0.40}{##1}}}
\expandafter\def\csname PY@tok@go\endcsname{\def\PY@tc##1{\textcolor[rgb]{0.53,0.53,0.53}{##1}}}
\expandafter\def\csname PY@tok@ge\endcsname{\let\PY@it=\textit}
\expandafter\def\csname PY@tok@vc\endcsname{\def\PY@tc##1{\textcolor[rgb]{0.10,0.09,0.49}{##1}}}
\expandafter\def\csname PY@tok@il\endcsname{\def\PY@tc##1{\textcolor[rgb]{0.40,0.40,0.40}{##1}}}
\expandafter\def\csname PY@tok@cs\endcsname{\let\PY@it=\textit\def\PY@tc##1{\textcolor[rgb]{0.25,0.50,0.50}{##1}}}
\expandafter\def\csname PY@tok@cp\endcsname{\def\PY@tc##1{\textcolor[rgb]{0.74,0.48,0.00}{##1}}}
\expandafter\def\csname PY@tok@gi\endcsname{\def\PY@tc##1{\textcolor[rgb]{0.00,0.63,0.00}{##1}}}
\expandafter\def\csname PY@tok@gh\endcsname{\let\PY@bf=\textbf\def\PY@tc##1{\textcolor[rgb]{0.00,0.00,0.50}{##1}}}
\expandafter\def\csname PY@tok@ni\endcsname{\let\PY@bf=\textbf\def\PY@tc##1{\textcolor[rgb]{0.60,0.60,0.60}{##1}}}
\expandafter\def\csname PY@tok@nl\endcsname{\def\PY@tc##1{\textcolor[rgb]{0.63,0.63,0.00}{##1}}}
\expandafter\def\csname PY@tok@nn\endcsname{\let\PY@bf=\textbf\def\PY@tc##1{\textcolor[rgb]{0.00,0.00,1.00}{##1}}}
\expandafter\def\csname PY@tok@no\endcsname{\def\PY@tc##1{\textcolor[rgb]{0.53,0.00,0.00}{##1}}}
\expandafter\def\csname PY@tok@na\endcsname{\def\PY@tc##1{\textcolor[rgb]{0.49,0.56,0.16}{##1}}}
\expandafter\def\csname PY@tok@nb\endcsname{\def\PY@tc##1{\textcolor[rgb]{0.00,0.50,0.00}{##1}}}
\expandafter\def\csname PY@tok@nc\endcsname{\let\PY@bf=\textbf\def\PY@tc##1{\textcolor[rgb]{0.00,0.00,1.00}{##1}}}
\expandafter\def\csname PY@tok@nd\endcsname{\def\PY@tc##1{\textcolor[rgb]{0.67,0.13,1.00}{##1}}}
\expandafter\def\csname PY@tok@ne\endcsname{\let\PY@bf=\textbf\def\PY@tc##1{\textcolor[rgb]{0.82,0.25,0.23}{##1}}}
\expandafter\def\csname PY@tok@nf\endcsname{\def\PY@tc##1{\textcolor[rgb]{0.00,0.00,1.00}{##1}}}
\expandafter\def\csname PY@tok@si\endcsname{\let\PY@bf=\textbf\def\PY@tc##1{\textcolor[rgb]{0.73,0.40,0.53}{##1}}}
\expandafter\def\csname PY@tok@s2\endcsname{\def\PY@tc##1{\textcolor[rgb]{0.73,0.13,0.13}{##1}}}
\expandafter\def\csname PY@tok@vi\endcsname{\def\PY@tc##1{\textcolor[rgb]{0.10,0.09,0.49}{##1}}}
\expandafter\def\csname PY@tok@nt\endcsname{\let\PY@bf=\textbf\def\PY@tc##1{\textcolor[rgb]{0.00,0.50,0.00}{##1}}}
\expandafter\def\csname PY@tok@nv\endcsname{\def\PY@tc##1{\textcolor[rgb]{0.10,0.09,0.49}{##1}}}
\expandafter\def\csname PY@tok@s1\endcsname{\def\PY@tc##1{\textcolor[rgb]{0.73,0.13,0.13}{##1}}}
\expandafter\def\csname PY@tok@kd\endcsname{\let\PY@bf=\textbf\def\PY@tc##1{\textcolor[rgb]{0.00,0.50,0.00}{##1}}}
\expandafter\def\csname PY@tok@sh\endcsname{\def\PY@tc##1{\textcolor[rgb]{0.73,0.13,0.13}{##1}}}
\expandafter\def\csname PY@tok@sc\endcsname{\def\PY@tc##1{\textcolor[rgb]{0.73,0.13,0.13}{##1}}}
\expandafter\def\csname PY@tok@sx\endcsname{\def\PY@tc##1{\textcolor[rgb]{0.00,0.50,0.00}{##1}}}
\expandafter\def\csname PY@tok@bp\endcsname{\def\PY@tc##1{\textcolor[rgb]{0.00,0.50,0.00}{##1}}}
\expandafter\def\csname PY@tok@c1\endcsname{\let\PY@it=\textit\def\PY@tc##1{\textcolor[rgb]{0.25,0.50,0.50}{##1}}}
\expandafter\def\csname PY@tok@kc\endcsname{\let\PY@bf=\textbf\def\PY@tc##1{\textcolor[rgb]{0.00,0.50,0.00}{##1}}}
\expandafter\def\csname PY@tok@c\endcsname{\let\PY@it=\textit\def\PY@tc##1{\textcolor[rgb]{0.25,0.50,0.50}{##1}}}
\expandafter\def\csname PY@tok@mf\endcsname{\def\PY@tc##1{\textcolor[rgb]{0.40,0.40,0.40}{##1}}}
\expandafter\def\csname PY@tok@err\endcsname{\def\PY@bc##1{\setlength{\fboxsep}{0pt}\fcolorbox[rgb]{1.00,0.00,0.00}{1,1,1}{\strut ##1}}}
\expandafter\def\csname PY@tok@mb\endcsname{\def\PY@tc##1{\textcolor[rgb]{0.40,0.40,0.40}{##1}}}
\expandafter\def\csname PY@tok@ss\endcsname{\def\PY@tc##1{\textcolor[rgb]{0.10,0.09,0.49}{##1}}}
\expandafter\def\csname PY@tok@sr\endcsname{\def\PY@tc##1{\textcolor[rgb]{0.73,0.40,0.53}{##1}}}
\expandafter\def\csname PY@tok@mo\endcsname{\def\PY@tc##1{\textcolor[rgb]{0.40,0.40,0.40}{##1}}}
\expandafter\def\csname PY@tok@kn\endcsname{\let\PY@bf=\textbf\def\PY@tc##1{\textcolor[rgb]{0.00,0.50,0.00}{##1}}}
\expandafter\def\csname PY@tok@mi\endcsname{\def\PY@tc##1{\textcolor[rgb]{0.40,0.40,0.40}{##1}}}
\expandafter\def\csname PY@tok@gp\endcsname{\let\PY@bf=\textbf\def\PY@tc##1{\textcolor[rgb]{0.00,0.00,0.50}{##1}}}
\expandafter\def\csname PY@tok@o\endcsname{\def\PY@tc##1{\textcolor[rgb]{0.40,0.40,0.40}{##1}}}
\expandafter\def\csname PY@tok@kr\endcsname{\let\PY@bf=\textbf\def\PY@tc##1{\textcolor[rgb]{0.00,0.50,0.00}{##1}}}
\expandafter\def\csname PY@tok@s\endcsname{\def\PY@tc##1{\textcolor[rgb]{0.73,0.13,0.13}{##1}}}
\expandafter\def\csname PY@tok@kp\endcsname{\def\PY@tc##1{\textcolor[rgb]{0.00,0.50,0.00}{##1}}}
\expandafter\def\csname PY@tok@w\endcsname{\def\PY@tc##1{\textcolor[rgb]{0.73,0.73,0.73}{##1}}}
\expandafter\def\csname PY@tok@kt\endcsname{\def\PY@tc##1{\textcolor[rgb]{0.69,0.00,0.25}{##1}}}
\expandafter\def\csname PY@tok@ow\endcsname{\let\PY@bf=\textbf\def\PY@tc##1{\textcolor[rgb]{0.67,0.13,1.00}{##1}}}
\expandafter\def\csname PY@tok@sb\endcsname{\def\PY@tc##1{\textcolor[rgb]{0.73,0.13,0.13}{##1}}}
\expandafter\def\csname PY@tok@k\endcsname{\let\PY@bf=\textbf\def\PY@tc##1{\textcolor[rgb]{0.00,0.50,0.00}{##1}}}
\expandafter\def\csname PY@tok@se\endcsname{\let\PY@bf=\textbf\def\PY@tc##1{\textcolor[rgb]{0.73,0.40,0.13}{##1}}}
\expandafter\def\csname PY@tok@sd\endcsname{\let\PY@it=\textit\def\PY@tc##1{\textcolor[rgb]{0.73,0.13,0.13}{##1}}}

\def\PYZbs{\char`\\}
\def\PYZus{\char`\_}
\def\PYZob{\char`\{}
\def\PYZcb{\char`\}}
\def\PYZca{\char`\^}
\def\PYZam{\char`\&}
\def\PYZlt{\char`\<}
\def\PYZgt{\char`\>}
\def\PYZsh{\char`\#}
\def\PYZpc{\char`\%}
\def\PYZdl{\char`\$}
\def\PYZhy{\char`\-}
\def\PYZsq{\char`\'}
\def\PYZdq{\char`\"}
\def\PYZti{\char`\~}
% for compatibility with earlier versions
\def\PYZat{@}
\def\PYZlb{[}
\def\PYZrb{]}
\makeatother


    % Exact colors from NB
    \definecolor{incolor}{rgb}{0.0, 0.0, 0.5}
    \definecolor{outcolor}{rgb}{0.545, 0.0, 0.0}



    
    % Prevent overflowing lines due to hard-to-break entities
    \sloppy 
    % Setup hyperref package
    \hypersetup{
      breaklinks=true,  % so long urls are correctly broken across lines
      colorlinks=true,
      urlcolor=blue,
      linkcolor=darkorange,
      citecolor=darkgreen,
      }
    % Slightly bigger margins than the latex defaults
    
    \geometry{verbose,tmargin=1in,bmargin=1in,lmargin=1in,rmargin=1in}
    
    

    \begin{document}
    
    
    \maketitle
    
    

    
    \#

EE 379K: Data Science Lab

\#

Lab 1 - 9/11/17

\#\#

Rachel Chen and Kevin Yee

\#\#\#

rjc2737 and kjy252

    \subsubsection{Question 1:}\label{question-1}

\begin{enumerate}
\def\labelenumi{\arabic{enumi}.}
\tightlist
\item
  Create 1000 samples from a Gaussian distribution with mean -10 and
  standard deviation 5. Create another 1000 samples from another
  independent Gaussian with mean 10 and standard deviation 5.
\end{enumerate}

\begin{enumerate}
\def\labelenumi{(\alph{enumi})}
\tightlist
\item
  Take the sum of 2 these Gaussians by adding the two sets of 1000
  points, point by point, and plot the histogram of the resulting 1000
  points. What do you observe?
\item
  Estimate the mean and the variance of the sum.
\end{enumerate}

    \begin{Verbatim}[commandchars=\\\{\}]
{\color{incolor}In [{\color{incolor}2}]:} \PY{o}{\PYZpc{}}\PY{k}{matplotlib} inline
        \PY{k+kn}{import} \PY{n+nn}{numpy} \PY{k+kn}{as} \PY{n+nn}{np}
        \PY{k+kn}{import} \PY{n+nn}{matplotlib.pyplot} \PY{k+kn}{as} \PY{n+nn}{plt}
        \PY{k+kn}{import} \PY{n+nn}{pandas} \PY{k+kn}{as} \PY{n+nn}{pd}
\end{Verbatim}

    \begin{Verbatim}[commandchars=\\\{\}]
{\color{incolor}In [{\color{incolor}3}]:} \PY{c}{\PYZsh{}Problem 1}
        \PY{n}{X} \PY{o}{=} \PY{n}{np}\PY{o}{.}\PY{n}{random}\PY{o}{.}\PY{n}{normal}\PY{p}{(}\PY{o}{\PYZhy{}}\PY{l+m+mi}{10}\PY{p}{,} \PY{l+m+mi}{5}\PY{p}{,} \PY{l+m+mi}{1000}\PY{p}{)}
        \PY{n}{Y} \PY{o}{=} \PY{n}{np}\PY{o}{.}\PY{n}{random}\PY{o}{.}\PY{n}{normal}\PY{p}{(}\PY{l+m+mi}{10}\PY{p}{,} \PY{l+m+mi}{5}\PY{p}{,} \PY{l+m+mi}{1000}\PY{p}{)}
        \PY{n}{Z} \PY{o}{=} \PY{n}{np}\PY{o}{.}\PY{n}{add}\PY{p}{(}\PY{n}{X}\PY{p}{,}\PY{n}{Y}\PY{p}{)}
        \PY{c}{\PYZsh{}https://stackoverflow.com/questions/38747612/whats\PYZhy{}the\PYZhy{}difference\PYZhy{}between\PYZhy{}numpy\PYZhy{}adda\PYZhy{}b\PYZhy{}and\PYZhy{}ab}
\end{Verbatim}

    \begin{Verbatim}[commandchars=\\\{\}]
{\color{incolor}In [{\color{incolor}4}]:} \PY{n}{plt}\PY{o}{.}\PY{n}{title}\PY{p}{(}\PY{l+s}{\PYZdq{}}\PY{l+s}{Addition of two Gaussian}\PY{l+s}{\PYZdq{}}\PY{p}{)}
        
        \PY{n}{plt}\PY{o}{.}\PY{n}{hist}\PY{p}{(}\PY{n}{X}\PY{p}{,} \PY{n}{label} \PY{o}{=}\PY{l+s}{\PYZsq{}}\PY{l+s}{X Distribution}\PY{l+s}{\PYZsq{}}\PY{p}{)}
        \PY{n}{plt}\PY{o}{.}\PY{n}{hist}\PY{p}{(}\PY{n}{Y}\PY{p}{,} \PY{n}{label} \PY{o}{=} \PY{l+s}{\PYZsq{}}\PY{l+s}{Y Distibution}\PY{l+s}{\PYZsq{}}\PY{p}{)}
        \PY{n}{plt}\PY{o}{.}\PY{n}{hist}\PY{p}{(}\PY{n}{Z}\PY{p}{,} \PY{n}{label} \PY{o}{=} \PY{l+s}{\PYZsq{}}\PY{l+s}{Z Distribution}\PY{l+s}{\PYZsq{}}\PY{p}{)}
        \PY{n}{plt}\PY{o}{.}\PY{n}{legend}\PY{p}{(}\PY{n}{loc} \PY{o}{=} \PY{l+s}{\PYZsq{}}\PY{l+s}{upper left}\PY{l+s}{\PYZsq{}}\PY{p}{)}
\end{Verbatim}

            \begin{Verbatim}[commandchars=\\\{\}]
{\color{outcolor}Out[{\color{outcolor}4}]:} <matplotlib.legend.Legend at 0x2136358>
\end{Verbatim}
        
    \begin{center}
    \adjustimage{max size={0.9\linewidth}{0.9\paperheight}}{Lab1_files/Lab1_4_1.png}
    \end{center}
    { \hspace*{\fill} \\}
    
    \begin{Verbatim}[commandchars=\\\{\}]
{\color{incolor}In [{\color{incolor}5}]:} \PY{k}{print} \PY{l+s}{\PYZdq{}}\PY{l+s}{The mean is}\PY{l+s}{\PYZdq{}}\PY{p}{,}
        \PY{k}{print}\PY{p}{(}\PY{l+s}{\PYZdq{}}\PY{l+s+si}{\PYZpc{}.2f}\PY{l+s}{\PYZdq{}} \PY{o}{\PYZpc{}} \PY{n}{np}\PY{o}{.}\PY{n}{mean}\PY{p}{(}\PY{n}{Z}\PY{p}{)}\PY{p}{)}
        \PY{k}{print} \PY{l+s}{\PYZdq{}}\PY{l+s}{The variance is}\PY{l+s}{\PYZdq{}}\PY{p}{,}
        \PY{k}{print}\PY{p}{(}\PY{l+s}{\PYZdq{}}\PY{l+s+si}{\PYZpc{}.2f}\PY{l+s}{\PYZdq{}} \PY{o}{\PYZpc{}} \PY{n}{np}\PY{o}{.}\PY{n}{var}\PY{p}{(}\PY{n}{Z}\PY{p}{)}\PY{p}{)}
\end{Verbatim}

    \begin{Verbatim}[commandchars=\\\{\}]
The mean is 0.27
The variance is 48.86
    \end{Verbatim}

    \paragraph{Problem 1 Observations:}\label{problem-1-observations}

The sum of 2 independent Gaussians results in a Gaussian with the sum of
the two means and sum of two variances. i.e.~given
\[X: N(\mu_x,\theta^2_x)\] \[Y: N(\mu_y, \theta^2_y)\] \[Z = X + Y\]

\[Z: N(\mu_x + \mu_y, \theta^2_x + \theta^2_y)\]

    \subsubsection{Question 2:}\label{question-2}

\begin{enumerate}
\def\labelenumi{\arabic{enumi}.}
\setcounter{enumi}{1}
\tightlist
\item
  Central Limit Theorem. Let Xi be an iid Bernoulli random variable with
  value \{-1,1\}. Look at the random variable Zn = 1 n PXi . By taking
  1000 draws from Zn, plot its histogram. Check that for small n (say,
  5-10) Zn does not look that much like a Gaussian, but when n is bigger
  (already by the time n = 30 or 50) it looks much more like a Gaussian.
  Check also for much bigger n: n = 250, to see that at this point, one
  can really see the bell curve.
\end{enumerate}

    \begin{Verbatim}[commandchars=\\\{\}]
{\color{incolor}In [{\color{incolor}6}]:} \PY{c}{\PYZsh{}Problem 2}
        
        \PY{c}{\PYZsh{}n\PYZhy{}trials}
        \PY{n}{numsamples} \PY{o}{=} \PY{p}{[}\PY{l+m+mi}{5}\PY{p}{,}\PY{l+m+mi}{50}\PY{p}{,}\PY{l+m+mi}{250}\PY{p}{,}\PY{l+m+mi}{500}\PY{p}{]} 
        \PY{n}{trials} \PY{o}{=} \PY{p}{[}\PY{p}{]}
        \PY{k}{for} \PY{n}{k} \PY{o+ow}{in} \PY{n+nb}{range}\PY{p}{(}\PY{n+nb}{len}\PY{p}{(}\PY{n}{numsamples}\PY{p}{)}\PY{p}{)}\PY{p}{:}
            \PY{k}{for} \PY{n}{j} \PY{o+ow}{in} \PY{n+nb}{range}\PY{p}{(}\PY{l+m+mi}{1000}\PY{p}{)}\PY{p}{:}
                \PY{n}{x} \PY{o}{=} \PY{n}{np}\PY{o}{.}\PY{n}{random}\PY{o}{.}\PY{n}{binomial}\PY{p}{(}\PY{l+m+mi}{1}\PY{p}{,} \PY{l+m+mf}{0.5}\PY{p}{,} \PY{n}{numsamples}\PY{p}{[}\PY{n}{k}\PY{p}{]}\PY{p}{)}\PY{p}{;} \PY{c}{\PYZsh{}bernoulli}
                \PY{k}{for} \PY{n}{i} \PY{o+ow}{in} \PY{n+nb}{range}\PY{p}{(}\PY{n+nb}{len}\PY{p}{(}\PY{n}{x}\PY{p}{)}\PY{p}{)}\PY{p}{:}
                    \PY{k}{if} \PY{n}{x}\PY{p}{[}\PY{n}{i}\PY{p}{]} \PY{o}{==} \PY{l+m+mi}{0}\PY{p}{:} \PY{n}{x}\PY{p}{[}\PY{n}{i}\PY{p}{]} \PY{o}{=} \PY{o}{\PYZhy{}}\PY{l+m+mi}{1} \PY{c}{\PYZsh{}spans \PYZhy{}1 to 1}
                \PY{n}{trials}\PY{o}{.}\PY{n}{append}\PY{p}{(}\PY{n+nb}{float}\PY{p}{(}\PY{n+nb}{sum}\PY{p}{(}\PY{n}{x}\PY{p}{)}\PY{p}{)}\PY{o}{/}\PY{n}{numsamples}\PY{p}{[}\PY{n}{k}\PY{p}{]}\PY{p}{)} \PY{c}{\PYZsh{}average of sum}
            \PY{n}{plt}\PY{o}{.}\PY{n}{figure}\PY{p}{(}\PY{n}{k}\PY{p}{)}
            \PY{n}{plt}\PY{o}{.}\PY{n}{title}\PY{p}{(}\PY{l+s}{\PYZdq{}}\PY{l+s}{Distribution with }\PY{l+s}{\PYZdq{}} \PY{o}{+} \PY{n+nb}{str}\PY{p}{(}\PY{n}{numsamples}\PY{p}{[}\PY{n}{k}\PY{p}{]}\PY{p}{)} \PY{o}{+} \PY{l+s}{\PYZdq{}}\PY{l+s}{ samples}\PY{l+s}{\PYZdq{}}\PY{p}{)}
            \PY{n}{plt}\PY{o}{.}\PY{n}{hist}\PY{p}{(}\PY{n}{trials}\PY{p}{,} \PY{n}{bins} \PY{o}{=} \PY{l+m+mi}{30}\PY{p}{)}
\end{Verbatim}

    \begin{center}
    \adjustimage{max size={0.9\linewidth}{0.9\paperheight}}{Lab1_files/Lab1_8_0.png}
    \end{center}
    { \hspace*{\fill} \\}
    
    \begin{center}
    \adjustimage{max size={0.9\linewidth}{0.9\paperheight}}{Lab1_files/Lab1_8_1.png}
    \end{center}
    { \hspace*{\fill} \\}
    
    \begin{center}
    \adjustimage{max size={0.9\linewidth}{0.9\paperheight}}{Lab1_files/Lab1_8_2.png}
    \end{center}
    { \hspace*{\fill} \\}
    
    \begin{center}
    \adjustimage{max size={0.9\linewidth}{0.9\paperheight}}{Lab1_files/Lab1_8_3.png}
    \end{center}
    { \hspace*{\fill} \\}
    
    \subsubsection{Question 3:}\label{question-3}

\begin{enumerate}
\def\labelenumi{\arabic{enumi}.}
\setcounter{enumi}{2}
\tightlist
\item
  Estimate the mean and standard deviation from 1 dimensional data:
  generate 25,000 samples from a Gaussian distribution with mean 0 and
  standard deviation 5. Then estimate the mean and standard deviation of
  this gaussian using elementary numpy commands, i.e., addition,
  multiplication, division (do not use a command that takes data and
  returns the mean or standard deviation).
\end{enumerate}

    \begin{Verbatim}[commandchars=\\\{\}]
{\color{incolor}In [{\color{incolor}7}]:} \PY{n}{k} \PY{o}{=} \PY{n}{np}\PY{o}{.}\PY{n}{random}\PY{o}{.}\PY{n}{normal}\PY{p}{(}\PY{l+m+mi}{0}\PY{p}{,} \PY{l+m+mi}{5}\PY{p}{,} \PY{l+m+mi}{25000}\PY{p}{)}
        \PY{n}{mean} \PY{o}{=} \PY{n}{k}\PY{o}{.}\PY{n}{sum}\PY{p}{(}\PY{p}{)}\PY{o}{/}\PY{n+nb}{len}\PY{p}{(}\PY{n}{k}\PY{p}{)}
        \PY{n}{std} \PY{o}{=} \PY{n}{np}\PY{o}{.}\PY{n}{sqrt}\PY{p}{(}\PY{n}{np}\PY{o}{.}\PY{n}{sum}\PY{p}{(}\PY{n}{np}\PY{o}{.}\PY{n}{square}\PY{p}{(}\PY{n}{k}\PY{o}{\PYZhy{}}\PY{n}{mean}\PY{p}{)}\PY{p}{)}\PY{o}{/}\PY{n+nb}{len}\PY{p}{(}\PY{n}{k}\PY{p}{)}\PY{p}{)}
        \PY{k}{print} \PY{n}{mean}
        \PY{k}{print} \PY{n}{std}
\end{Verbatim}

    \begin{Verbatim}[commandchars=\\\{\}]
0.0149204618984
5.00275383556
    \end{Verbatim}

    \subsubsection{Question 4:}\label{question-4}

\begin{enumerate}
\def\labelenumi{\arabic{enumi}.}
\setcounter{enumi}{3}
\tightlist
\item
  Estimate the mean and covariance matrix for multi-dimensional data:
  generate 10,000 samples of 2 dimensional data from the Gaussian
  distribution Then, estimate the mean and covariance matrix for this
  multi-dimensional data using elementary numpy commands, i.e.,
  addition, multiplication, division (do not use a command that takes
  data and returns the mean or standard deviation).
\end{enumerate}

    \begin{Verbatim}[commandchars=\\\{\}]
{\color{incolor}In [{\color{incolor}8}]:} \PY{n}{mean} \PY{o}{=} \PY{p}{[}\PY{o}{\PYZhy{}}\PY{l+m+mi}{5}\PY{p}{,} \PY{l+m+mi}{5}\PY{p}{]}
        \PY{n}{cov} \PY{o}{=} \PY{p}{[}\PY{p}{[}\PY{l+m+mi}{20}\PY{p}{,} \PY{o}{.}\PY{l+m+mi}{8}\PY{p}{]}\PY{p}{,} \PY{p}{[}\PY{o}{.}\PY{l+m+mi}{8}\PY{p}{,} \PY{l+m+mi}{30}\PY{p}{]}\PY{p}{]}
        \PY{n}{x} \PY{o}{=} \PY{n}{np}\PY{o}{.}\PY{n}{random}\PY{o}{.}\PY{n}{multivariate\PYZus{}normal}\PY{p}{(}\PY{n}{mean}\PY{p}{,} \PY{n}{cov}\PY{p}{,} \PY{l+m+mi}{10000}\PY{p}{)}
        \PY{k}{print} \PY{n}{x}
        \PY{n}{mean} \PY{o}{=} \PY{n}{np}\PY{o}{.}\PY{n}{sum}\PY{p}{(}\PY{n}{x}\PY{p}{,} \PY{n}{axis} \PY{o}{=} \PY{l+m+mi}{0}\PY{p}{)}\PY{o}{/}\PY{n+nb}{len}\PY{p}{(}\PY{n}{x}\PY{p}{)}
        
        \PY{k}{print} \PY{l+s}{\PYZdq{}}\PY{l+s}{The mean matrix is}\PY{l+s}{\PYZdq{}}\PY{p}{,}
        \PY{k}{print} \PY{n}{mean}
        
        \PY{c}{\PYZsh{}https://stackoverflow.com/questions/27448352/how\PYZhy{}numpy\PYZhy{}cov\PYZhy{}function\PYZhy{}is\PYZhy{}implemented}
\end{Verbatim}

    \begin{Verbatim}[commandchars=\\\{\}]
[[ -8.37707136   8.48436319]
 [ -7.13527547   4.65723797]
 [ -0.04755268  10.57112753]
 {\ldots}, 
 [ -4.25173414   7.16760325]
 [-10.03460056   9.46320271]
 [ -8.60942973   0.85905407]]
The mean matrix is [-4.98667696  4.97735342]
    \end{Verbatim}

    For reference, the Covariance matrix is defined below:
\includegraphics{http://i.markdownnotes.com/image_EOn9wA1.png}

    \begin{Verbatim}[commandchars=\\\{\}]
{\color{incolor}In [{\color{incolor}9}]:} \PY{n}{varX} \PY{o}{=} \PY{n}{np}\PY{o}{.}\PY{n}{sum}\PY{p}{(}\PY{n}{np}\PY{o}{.}\PY{n}{square}\PY{p}{(}\PY{n}{x}\PY{p}{[}\PY{p}{:}\PY{p}{,}\PY{l+m+mi}{0}\PY{p}{]}\PY{o}{\PYZhy{}}\PY{n}{mean}\PY{p}{[}\PY{l+m+mi}{0}\PY{p}{]}\PY{p}{)}\PY{p}{)}\PY{o}{/}\PY{n+nb}{len}\PY{p}{(}\PY{n}{x}\PY{p}{)}
        \PY{n}{varY} \PY{o}{=} \PY{n}{np}\PY{o}{.}\PY{n}{sum}\PY{p}{(}\PY{n}{np}\PY{o}{.}\PY{n}{square}\PY{p}{(}\PY{n}{x}\PY{p}{[}\PY{p}{:}\PY{p}{,}\PY{l+m+mi}{1}\PY{p}{]}\PY{o}{\PYZhy{}}\PY{n}{mean}\PY{p}{[}\PY{l+m+mi}{1}\PY{p}{]}\PY{p}{)}\PY{p}{)}\PY{o}{/}\PY{n+nb}{len}\PY{p}{(}\PY{n}{x}\PY{p}{)}
        \PY{n}{CovXY} \PY{o}{=} \PY{n}{np}\PY{o}{.}\PY{n}{sum}\PY{p}{(}\PY{p}{(}\PY{n}{x}\PY{p}{[}\PY{p}{:}\PY{p}{,}\PY{l+m+mi}{0}\PY{p}{]}\PY{o}{\PYZhy{}}\PY{n}{mean}\PY{p}{[}\PY{l+m+mi}{0}\PY{p}{]}\PY{p}{)} \PY{o}{*} \PY{p}{(}\PY{n}{x}\PY{p}{[}\PY{p}{:}\PY{p}{,}\PY{l+m+mi}{1}\PY{p}{]}\PY{o}{\PYZhy{}}\PY{n}{mean}\PY{p}{[}\PY{l+m+mi}{1}\PY{p}{]}\PY{p}{)}\PY{p}{)}\PY{o}{/}\PY{n+nb}{len}\PY{p}{(}\PY{n}{x}\PY{p}{)}
        
        \PY{k}{print} \PY{l+s}{\PYZdq{}}\PY{l+s}{The covariance matrix is}\PY{l+s}{\PYZdq{}}
        \PY{k}{print} \PY{n}{np}\PY{o}{.}\PY{n}{array}\PY{p}{(}\PY{p}{[}\PY{p}{[}\PY{n}{varX}\PY{p}{,} \PY{n}{CovXY}\PY{p}{]}\PY{p}{,} \PY{p}{[}\PY{n}{CovXY}\PY{p}{,} \PY{n}{varY}\PY{p}{]}\PY{p}{]}\PY{p}{)}
\end{Verbatim}

    \begin{Verbatim}[commandchars=\\\{\}]
The covariance matrix is
[[ 19.8701742    0.43441176]
 [  0.43441176  29.58796425]]
    \end{Verbatim}

    \subsubsection{Question 5:}\label{question-5}

Each row is a patient and the last column is the condition that the
patient has. Do data exploration using Pandas and other visualization
tools to understand what you can about the dataset. For example:

\subparagraph{(a) How many patients and how many features are
there?}\label{a-how-many-patients-and-how-many-features-are-there}

\begin{verbatim}
452 patients.
280 features.
\end{verbatim}

\subparagraph{(b) What is the meaning of the first 4 features? See if
you can understand what they
mean.}\label{b-what-is-the-meaning-of-the-first-4-features-see-if-you-can-understand-what-they-mean.}

\begin{verbatim}
Feature 1: Age
Feature 2: Sex
Feature 3: Systolic blood presure
Feature 4: Diastolic blood presssure
\end{verbatim}

    \begin{Verbatim}[commandchars=\\\{\}]
{\color{incolor}In [{\color{incolor}10}]:} \PY{n}{df0} \PY{o}{=} \PY{n}{pd}\PY{o}{.}\PY{n}{read\PYZus{}csv}\PY{p}{(}\PY{l+s}{\PYZsq{}}\PY{l+s}{PatientData.csv}\PY{l+s}{\PYZsq{}}\PY{p}{)}
         \PY{n}{colNames} \PY{o}{=} \PY{p}{[}\PY{p}{]}\PY{p}{;}
         \PY{c}{\PYZsh{}define column names}
         \PY{k}{for} \PY{n}{i} \PY{o+ow}{in} \PY{n+nb}{range}\PY{p}{(}\PY{n+nb}{len}\PY{p}{(}\PY{n}{df0}\PY{o}{.}\PY{n}{columns}\PY{p}{)}\PY{p}{)}\PY{p}{:}
             \PY{n}{colNames}\PY{o}{.}\PY{n}{append}\PY{p}{(}\PY{l+s}{\PYZdq{}}\PY{l+s}{Feature }\PY{l+s}{\PYZdq{}} \PY{o}{+} \PY{n+nb}{str}\PY{p}{(}\PY{n}{i}\PY{p}{)}\PY{p}{)}
         \PY{c}{\PYZsh{}import into pandas dataframe    }
         \PY{n}{df0} \PY{o}{=} \PY{n}{pd}\PY{o}{.}\PY{n}{read\PYZus{}csv}\PY{p}{(}\PY{l+s}{\PYZsq{}}\PY{l+s}{PatientData.csv}\PY{l+s}{\PYZsq{}}\PY{p}{,}\PY{n}{names} \PY{o}{=} \PY{n}{colNames}\PY{p}{)}
         
         \PY{c}{\PYZsh{}take the average \PYZdq{}blood pressure\PYZdq{} for each age and plot}
         \PY{n}{dfpiv} \PY{o}{=} \PY{n}{pd}\PY{o}{.}\PY{n}{pivot\PYZus{}table}\PY{p}{(}\PY{n}{df0}\PY{p}{,}\PY{n}{index}\PY{o}{=}\PY{p}{[}\PY{l+s}{\PYZsq{}}\PY{l+s}{Feature 0}\PY{l+s}{\PYZsq{}}\PY{p}{]}\PY{p}{,} \PY{n}{values} \PY{o}{=} \PY{p}{[}\PY{l+s}{\PYZsq{}}\PY{l+s}{Feature 2}\PY{l+s}{\PYZsq{}}\PY{p}{]}\PY{p}{,} \PY{n}{aggfunc}\PY{o}{=}\PY{p}{[}\PY{n}{np}\PY{o}{.}\PY{n}{mean}\PY{p}{]}\PY{p}{)}
         \PY{n}{dfpiv}\PY{o}{.}\PY{n}{plot}\PY{p}{(}\PY{n}{kind} \PY{o}{=} \PY{l+s}{\PYZsq{}}\PY{l+s}{bar}\PY{l+s}{\PYZsq{}}\PY{p}{)}
         \PY{n}{dfpiv} \PY{o}{=} \PY{n}{pd}\PY{o}{.}\PY{n}{pivot\PYZus{}table}\PY{p}{(}\PY{n}{df0}\PY{p}{,}\PY{n}{index}\PY{o}{=}\PY{p}{[}\PY{l+s}{\PYZsq{}}\PY{l+s}{Feature 0}\PY{l+s}{\PYZsq{}}\PY{p}{]}\PY{p}{,} \PY{n}{values} \PY{o}{=} \PY{p}{[}\PY{l+s}{\PYZsq{}}\PY{l+s}{Feature 3}\PY{l+s}{\PYZsq{}}\PY{p}{]}\PY{p}{,} \PY{n}{aggfunc}\PY{o}{=}\PY{p}{[}\PY{n}{np}\PY{o}{.}\PY{n}{mean}\PY{p}{]}\PY{p}{)}
         \PY{n}{dfpiv}\PY{o}{.}\PY{n}{plot}\PY{p}{(}\PY{n}{kind} \PY{o}{=} \PY{l+s}{\PYZsq{}}\PY{l+s}{bar}\PY{l+s}{\PYZsq{}}\PY{p}{)}
\end{Verbatim}

            \begin{Verbatim}[commandchars=\\\{\}]
{\color{outcolor}Out[{\color{outcolor}10}]:} <matplotlib.axes.\_subplots.AxesSubplot at 0xaec3f28>
\end{Verbatim}
        
    \begin{center}
    \adjustimage{max size={0.9\linewidth}{0.9\paperheight}}{Lab1_files/Lab1_16_1.png}
    \end{center}
    { \hspace*{\fill} \\}
    
    \begin{center}
    \adjustimage{max size={0.9\linewidth}{0.9\paperheight}}{Lab1_files/Lab1_16_2.png}
    \end{center}
    { \hspace*{\fill} \\}
    
    Looking at the bar graph above, the x axis is of age and y axis is of
Feature 3. We can infer that Feature 3 correlates to blood pressure.
Younger people \textless{}10 have lower blood pressure and older people
have higher blood pressure.

Our observation is made in regards to the normal healthy 120/80
measurement for blood pressure

    \paragraph{(c) Are there missing values? Replace them with the average
of the corresponding feature
column}\label{c-are-there-missing-values-replace-them-with-the-average-of-the-corresponding-feature-column}

Yes. We can verify that we correctly replaced all the missing values by
looking at Column 13, where originally many values were missing. These
have been replaced with the average of the corresponding feature column

    \begin{Verbatim}[commandchars=\\\{\}]
{\color{incolor}In [{\color{incolor}11}]:} \PY{c}{\PYZsh{}replace ? with nan so the mean will ignore nan}
         \PY{n}{newdata} \PY{o}{=} \PY{n}{df0}\PY{o}{.}\PY{n}{replace}\PY{p}{(}\PY{l+s}{\PYZsq{}}\PY{l+s}{?}\PY{l+s}{\PYZsq{}}\PY{p}{,}\PY{n}{np}\PY{o}{.}\PY{n}{nan}\PY{p}{)} 
         
         \PY{c}{\PYZsh{}replace all missing values by iterating through columns}
         \PY{k}{for} \PY{n}{col} \PY{o+ow}{in} \PY{n}{newdata}\PY{o}{.}\PY{n}{columns}\PY{p}{:}
             \PY{n}{newdata}\PY{p}{[}\PY{n}{col}\PY{p}{]} \PY{o}{=} \PY{n}{newdata}\PY{p}{[}\PY{n}{col}\PY{p}{]}\PY{o}{.}\PY{n}{fillna}\PY{p}{(}\PY{n}{newdata}\PY{p}{[}\PY{n}{col}\PY{p}{]}\PY{o}{.}\PY{n}{map}\PY{p}{(}\PY{n+nb}{float}\PY{p}{)}\PY{o}{.}\PY{n}{mean}\PY{p}{(}\PY{p}{)}\PY{p}{)}
         
         \PY{c}{\PYZsh{}print to show soln is correct}
         \PY{n}{newdata}\PY{o}{.}\PY{n}{ix}\PY{p}{[}\PY{p}{:}\PY{p}{,}\PY{l+s}{\PYZsq{}}\PY{l+s}{Feature 13}\PY{l+s}{\PYZsq{}}\PY{p}{:}\PY{p}{]}\PY{o}{.}\PY{n}{head}\PY{p}{(}\PY{p}{)}
\end{Verbatim}

            \begin{Verbatim}[commandchars=\\\{\}]
{\color{outcolor}Out[{\color{outcolor}11}]:}   Feature 13 Feature 14  Feature 15  Feature 16  Feature 17  Feature 18  \textbackslash{}
         0   -13.5921         63           0          52          44           0   
         1   -13.5921         53           0          48           0           0   
         2         23         75           0          40          80           0   
         3   -13.5921         71           0          72          20           0   
         4   -13.5921    74.4634           0          48          40           0   
         
            Feature 19  Feature 20  Feature 21  Feature 22     {\ldots}       Feature 270  \textbackslash{}
         0           0          32           0           0     {\ldots}               0.0   
         1           0          24           0           0     {\ldots}               0.0   
         2           0          24           0           0     {\ldots}               0.0   
         3           0          48           0           0     {\ldots}               0.0   
         4           0          28           0           0     {\ldots}               0.0   
         
            Feature 271  Feature 272  Feature 273  Feature 274  Feature 275  \textbackslash{}
         0          9.0         -0.9          0.0          0.0          0.9   
         1          8.5          0.0          0.0          0.0          0.2   
         2          9.5         -2.4          0.0          0.0          0.3   
         3         12.2         -2.2          0.0          0.0          0.4   
         4         13.1         -3.6          0.0          0.0         -0.1   
         
            Feature 276  Feature 277  Feature 278  Feature 279  
         0          2.9         23.3         49.4            8  
         1          2.1         20.4         38.8            6  
         2          3.4         12.3         49.0           10  
         3          2.6         34.6         61.6            1  
         4          3.9         25.4         62.8            7  
         
         [5 rows x 267 columns]
\end{Verbatim}
        
    \paragraph{(d) How could you test which features strongly influence the
patient condition and which do
not?}\label{d-how-could-you-test-which-features-strongly-influence-the-patient-condition-and-which-do-not}

\textbf{Method 1} 1. Create a correlation matrix using pandas .corr()
function 2. Compare the correlation coefficient of all other columns in
comparison to the ``patient's condition column'' (Last column) 3. Sort
the data in descending order. The columns with the highest correlation
coefficient will be most related to the patient's condition. (This is
demonstrated in the code below)

\textbf{Method 2}: Because there are 280 features. We run into the
`curse of dimensionality.' We could alternatively run PCA (Principal
Component Analysis) on our dataset to determine which features minimizes
the mean squared distance between the original dataset to find out which
features are indeed most important.

PCA is probably the most effective, however, considering this is Data
Science Lab 1, I will tackle it with Method 1 (naiive solution).

    \paragraph{List what you think are the three most important
features.}\label{list-what-you-think-are-the-three-most-important-features.}

The three most important features are

\textbf{Feature 90}

\textbf{Feature 4}

\textbf{Feature 92}

(Data shown below)

    \begin{Verbatim}[commandchars=\\\{\}]
{\color{incolor}In [{\color{incolor}14}]:} \PY{n}{corr\PYZus{}matrix} \PY{o}{=} \PY{n}{newdata}\PY{o}{.}\PY{n}{corr}\PY{p}{(}\PY{p}{)}
         \PY{n}{corr\PYZus{}matrix}\PY{o}{.}\PY{n}{iloc}\PY{p}{[}\PY{p}{:}\PY{p}{,}\PY{o}{\PYZhy{}}\PY{l+m+mi}{1}\PY{p}{]}\PY{o}{.}\PY{n}{sort\PYZus{}values}\PY{p}{(}\PY{n}{ascending}\PY{o}{=}\PY{n+nb+bp}{False}\PY{p}{)}\PY{o}{.}\PY{n}{head}\PY{p}{(}\PY{p}{)}
\end{Verbatim}

            \begin{Verbatim}[commandchars=\\\{\}]
{\color{outcolor}Out[{\color{outcolor}14}]:} Feature 279    1.000000
         Feature 90     0.368876
         Feature 4      0.323879
         Feature 92     0.313982
         Feature 102    0.282523
         Name: Feature 279, dtype: float64
\end{Verbatim}
        
    \section{Written Questions}\label{written-questions}

    \begin{figure}
\centering
\includegraphics{http://i.markdownnotes.com/image_NAAlIyn.png}
\caption{}
\end{figure}

    \paragraph{1(a):}\label{a}

Marginal pmf from joint pmf: \[P(X = x) = \sum_y P(X=x,Y=y)\]
\[\therefore P(X = 1) = \frac{1}{4} + \frac{1}{3} = \frac{7}{12}\]

    \paragraph{1(b):}\label{b}

Conditional Probability: \[P(X | Y) = \frac{P(X, Y)}{P(Y)}\]
\[\therefore P(X=1 | Y = 1) = \frac{1/3}{1/6 + 1/3}\]
\[P(X=1 | Y = 1) = \frac{2}{3}\]

    \paragraph{1(c):}\label{c}

Bernoulli Equation \[P(X = 1) = \frac7{12}\] \[P(X = 0 ) = \frac5{12}\]
\[\therefore pq = \frac7{12} * \frac5{12} = \frac{35}{72}\]

    \paragraph{1(d):}\label{d}

Bernoulli Equation \[P(X = 1 | Y =1) = \frac2{3}\]
\[P(X = 1 | Y = 0 ) = \frac3{7}\]
\[\therefore pq = \frac2{3} * \frac1{3} = \frac{2}{9}\]

    \paragraph{1(e):}\label{e}

\[E[X^3 + X^2 + 3Y^7 | Y =1]\]
\[ = E[X^3 | Y = 1] + E[X^2 | Y = 1] + 3 * E[Y^7 | Y = 1] \]
\[ = \frac{1}{3} + \frac{1}{3} + 3 * 1\] \[ = 3\frac{2}{3}\]

    \begin{figure}
\centering
\includegraphics{http://i.markdownnotes.com/image_sgMTWJI.png}
\caption{}
\end{figure}

    Let W be defined as the \[Span~(v_1, v_2)\]

    \begin{Verbatim}[commandchars=\\\{\}]
{\color{incolor}In [{\color{incolor} }]:} \PY{k}{def} \PY{n+nf}{orthog}\PY{p}{(}\PY{n}{v1}\PY{p}{,} \PY{n}{v2}\PY{p}{,} \PY{n}{point}\PY{p}{)}\PY{p}{:}
            \PY{n}{u1} \PY{o}{=} \PY{p}{(}\PY{n}{np}\PY{o}{.}\PY{n}{dot}\PY{p}{(}\PY{n}{v1}\PY{p}{,}\PY{n}{point}\PY{p}{)}\PY{o}{/}\PY{n}{np}\PY{o}{.}\PY{n}{dot}\PY{p}{(}\PY{n}{v1}\PY{p}{,}\PY{n}{v1}\PY{p}{)}\PY{p}{)}\PY{o}{*}\PY{n}{v1}\PY{p}{;}
            \PY{n}{u2} \PY{o}{=} \PY{p}{(}\PY{n}{np}\PY{o}{.}\PY{n}{dot}\PY{p}{(}\PY{n}{v2}\PY{p}{,}\PY{n}{point}\PY{p}{)}\PY{p}{)}\PY{o}{/}\PY{n}{np}\PY{o}{.}\PY{n}{dot}\PY{p}{(}\PY{n}{v2}\PY{p}{,}\PY{n}{v2}\PY{p}{)}\PY{o}{*}\PY{n}{v2}\PY{p}{;}
            \PY{k}{return} \PY{n}{u1} \PY{o}{+} \PY{n}{u2}\PY{p}{;}
\end{Verbatim}

    \begin{Verbatim}[commandchars=\\\{\}]
{\color{incolor}In [{\color{incolor} }]:} \PY{n}{v1} \PY{o}{=} \PY{p}{[}\PY{l+m+mi}{1}\PY{p}{,} \PY{l+m+mi}{1}\PY{p}{,} \PY{l+m+mi}{1}\PY{p}{]}\PY{p}{;}
        \PY{n}{v2} \PY{o}{=} \PY{p}{[}\PY{l+m+mi}{1}\PY{p}{,} \PY{l+m+mi}{0}\PY{p}{,} \PY{l+m+mi}{0}\PY{p}{]}\PY{p}{;}
        \PY{n}{P1} \PY{o}{=} \PY{p}{[}\PY{l+m+mi}{3}\PY{p}{,} \PY{l+m+mi}{3}\PY{p}{,} \PY{l+m+mi}{3}\PY{p}{]}\PY{p}{;}
        \PY{n}{P2} \PY{o}{=} \PY{p}{[}\PY{l+m+mi}{1}\PY{p}{,} \PY{l+m+mi}{2}\PY{p}{,} \PY{l+m+mi}{3}\PY{p}{]}\PY{p}{;}
        \PY{n}{P3} \PY{o}{=} \PY{p}{[}\PY{l+m+mi}{0}\PY{p}{,} \PY{l+m+mi}{0}\PY{p}{,} \PY{l+m+mi}{1}\PY{p}{]}\PY{p}{;}
        
        \PY{k}{print}\PY{p}{(}\PY{l+s}{\PYZdq{}}\PY{l+s}{Point 1 projection}\PY{l+s}{\PYZdq{}}\PY{p}{,} \PY{n}{orthog}\PY{p}{(}\PY{n}{np}\PY{o}{.}\PY{n}{transpose}\PY{p}{(}\PY{n}{v1}\PY{p}{)}\PY{p}{,}\PY{n}{np}\PY{o}{.}\PY{n}{transpose}\PY{p}{(}\PY{n}{v2}\PY{p}{)}\PY{p}{,}\PY{n}{np}\PY{o}{.}\PY{n}{transpose}\PY{p}{(}\PY{n}{P1}\PY{p}{)}\PY{p}{)}\PY{p}{)}
        \PY{k}{print}\PY{p}{(}\PY{l+s}{\PYZdq{}}\PY{l+s}{Point 2 projection}\PY{l+s}{\PYZdq{}}\PY{p}{,} \PY{n}{orthog}\PY{p}{(}\PY{n}{np}\PY{o}{.}\PY{n}{transpose}\PY{p}{(}\PY{n}{v1}\PY{p}{)}\PY{p}{,}\PY{n}{np}\PY{o}{.}\PY{n}{transpose}\PY{p}{(}\PY{n}{v2}\PY{p}{)}\PY{p}{,}\PY{n}{np}\PY{o}{.}\PY{n}{transpose}\PY{p}{(}\PY{n}{P2}\PY{p}{)}\PY{p}{)}\PY{p}{)}
        \PY{k}{print}\PY{p}{(}\PY{l+s}{\PYZdq{}}\PY{l+s}{Point 3 projection}\PY{l+s}{\PYZdq{}}\PY{p}{,} \PY{n}{orthog}\PY{p}{(}\PY{n}{np}\PY{o}{.}\PY{n}{transpose}\PY{p}{(}\PY{n}{v1}\PY{p}{)}\PY{p}{,}\PY{n}{np}\PY{o}{.}\PY{n}{transpose}\PY{p}{(}\PY{n}{v2}\PY{p}{)}\PY{p}{,}\PY{n}{np}\PY{o}{.}\PY{n}{transpose}\PY{p}{(}\PY{n}{P3}\PY{p}{)}\PY{p}{)}\PY{p}{)}
\end{Verbatim}

    \begin{figure}
\centering
\includegraphics{http://i.markdownnotes.com/image_5gmzI2o.png}
\caption{}
\end{figure}

    Central Limit Theorem: \[\frac{S_n - n*E[X]}{\sqrt{n}\theta}\]
\[P(Head) = \frac{2}{3}\]vb\\
Let \[X_i\] be the random variable that a head is flipped. where X = 1
if heads and 0 otherwise. Then, let \[S_n = X_1 + X_2 +... X_n\]
Variance = \[\frac{2}{3} * \frac{1}{3}\] Using CTL:
\[\frac{50 - (100 * \frac{2}{3})}{\sqrt{100} * \sqrt{2/9}}\]
\[P(Z_{100} < -3.54) = \phi(-3.54)\]

    \begin{Verbatim}[commandchars=\\\{\}]
{\color{incolor}In [{\color{incolor} }]:} \PY{k+kn}{import} \PY{n+nn}{scipy.stats}
        \PY{n}{scipy}\PY{o}{.}\PY{n}{stats}\PY{o}{.}\PY{n}{norm}\PY{o}{.}\PY{n}{cdf}\PY{p}{(}\PY{o}{\PYZhy{}}\PY{l+m+mf}{3.54}\PY{p}{)}
\end{Verbatim}

    \[P(Z_{100} < -3.54) = 0.0002\] Probability of getting fewer than 50
heads is .02\%

    \begin{Verbatim}[commandchars=\\\{\}]
{\color{incolor}In [{\color{incolor} }]:} 
\end{Verbatim}

    \begin{Verbatim}[commandchars=\\\{\}]
{\color{incolor}In [{\color{incolor} }]:} 
\end{Verbatim}


    % Add a bibliography block to the postdoc
    
    
    
    \end{document}
